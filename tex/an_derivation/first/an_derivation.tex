\documentclass[12pt]{article}
\usepackage{latexsym}
\usepackage[margin=2.54cm, lmargin=2.54cm]{geometry}
\usepackage{amsmath}
\usepackage{amsfonts}
\usepackage{graphicx}
\usepackage[latin1]{inputenc}
\usepackage{calrsfs}
\usepackage{color}
\usepackage{cancel}
\usepackage{amssymb}

%\newcommand{\bibfont}{\footnotesize}

\linespread{1.5}


\newcommand{\vect}[1]{\boldsymbol{\mathrm{#1}}}
\newcommand{\mat}[1]{\boldsymbol{\mathrm{#1}}}
%\newcommand{\vect}[1]{\underline{\mathrm{#1}}}
%\newcommand{\mat}[1]{\underline{\underline{\mathrm{#1}}}}
\newcommand{\MSE}{\mathrm{MSE}}
\newcommand{\tr}{\text{tr}}
\newcommand{\diag}{\text{diag}}
\newcommand{\vecop}{\text{vec}}

\newcommand{\CP}{L}

%%%%%%%%%%%%%%%%%%%%%%%%%%%%%%%%%%%%%%%%%%%%%%%%%%%%%%%%%%%%%%%%%%%%%%%%%%%%%%%%%%%%%%%%%%%%%%%%%%%%%%%%%%%%%%% begin[document]
\begin{document}
\title{Comments conference paper}
\author{Fran\c cois Rottenberg}
\maketitle
%%%%%%%%%%%%%%%%%%%%%%%%%%%%%%%%%%%%%%%%%%%%%%%%%%%%%%%%%%%%%%%%%%%%%%%%%%%%%%%%%%%%%%%%%%%%%%%%%%%%%%%%%%%%% abstract
\large
\textcolor{red}{In my simulation, I consider that: $\mathbb{E}\left(\vect{w}\vect{w}^H\right) = \sigma^2_{an} \mat{I}_Q$, where $\sigma^2_{an} = 1/U$ is the AN autocorrelation.  \textcolor{blue}{Could you simulate for $\mathbb{E}\left({\vect{w}}'({\vect{w}}')^H\right)=\mat{I}$ please? so that it is in accordance with my derivations}\\
Then, I simulate the energy of the AN at Eve for a particular symbol $n$, i.e., one of the component of the signal $\vect{v}$. Each component of $\vect{v}$ is made from a summation of $U$ subcarriers thanks to the despreading matrix $\mat{S}^H$}\\

The AN at Eve is
\begin{align*}
	\vect{v}&=	\mat{S}^H \mat{H}_B |\mat{H}_E|^2 \vect{w}\\
	&=\mat{A} |\mat{H}_E|^2 \mat{V}_2 \vect{w}'\\
	&= \mat{U} \begin{pmatrix}
	\mat{\Sigma} & \mat{0}_{N-Q\times N}
	\end{pmatrix}  \begin{pmatrix}
	\mat{V}_1^H\\
	\mat{V}_2^H
	\end{pmatrix} |\mat{H}_E|^2 \mat{V}_2 \vect{w}'\\
	&=\mat{U} \mat{\Sigma}\mat{V}_1^H |\mat{H}_E|^2 \mat{V}_2 \vect{w}'
\end{align*}
\textcolor{red}{Therefore, since $\vect{w} = \mat{V}_2 \vect{w}'$, we have: $\mathbb{E}\left(\vect{w}\vect{w}^H\right) = \mat{V}_2 \mat{V}_2^H =  \sigma^2_{an} \mat{I}_Q$ } \textcolor{blue}{this is wrong! $\mat{V}_2 \mat{V}_2^H \neq   \sigma^2_{an} \mat{I}_Q$ because $\mat{V}_2$ is tall} \\
Let us now look at the covariance matrix
\begin{align*}
	\mathbb{E}\left(\vect{v}\vect{v}^H\right)&=\mathbb{E}\left(\mat{U} \mat{\Sigma}\mat{V}_1^H |\mat{H}_E|^2 \mat{V}_2 \vect{w}'\left(\mat{U} \mat{\Sigma}\mat{V}_1^H |\mat{H}_E|^2 \mat{V}_2 \vect{w}'\right)^H\right)\\
	&=\mathbb{E}\left(\mat{U} \mat{\Sigma}\mat{V}_1^H |\mat{H}_E|^2 \mat{V}_2 \vect{w}'\vect{w}'^H\mat{V}_2^H|\mat{H}_E|^2\mat{V}_1 \mat{\Sigma}^H   \mat{U}^H\right)
\end{align*}
Note that $\vect{w}'$ is independent of other random variable and has a unit covariance matrix. We can thus put the expectation inside to get
\begin{align*}
\mathbb{E}\left(\vect{v}\vect{v}^H\right)&=\mathbb{E}\left(\mat{U} \mat{\Sigma}\mat{V}_1^H |\mat{H}_E|^2 \mat{V}_2 \mat{V}_2^H|\mat{H}_E|^2\mat{V}_1 \mat{\Sigma}^H   \mat{U}^H\right)
\end{align*}
We rewrite $|\mat{H}_E|^2=\sum_{q=1}^Q|H_{E,q}|^2 \vect{e}_q \vect{e}_q^T $ where $\vect{e}_q$ is an all zero vector except a $1$ at row $q$ to isolate the independent random variable $H_E$
\begin{align*}
\mathbb{E}\left(\vect{v}\vect{v}^H\right)&=\sum_{q=1}^Q\sum_{q'=1}^Q\mathbb{E}(|H_{E,q}|^2|H_{E,q'}|^2)\mathbb{E}\left(\mat{U} \mat{\Sigma}\mat{V}_1^H  \vect{e}_q \vect{e}_q^T \mat{V}_2 \mat{V}_2^H \vect{e}_{q'} \vect{e}_{q'}^T\mat{V}_1 \mat{\Sigma}^H   \mat{U}^H\right)\\
&=\sum_{q=1}^Q\mathbb{E}(|H_{E,q}|^4)\mathbb{E}\left(\mat{U} \mat{\Sigma}\mat{V}_1^H  \vect{e}_q \vect{e}_q^T \mat{V}_2 \mat{V}_2^H \vect{e}_{q} \vect{e}_{q}^T\mat{V}_1 \mat{\Sigma}^H   \mat{U}^H\right)\\
&+\sum_{q=1}^Q\sum_{q'\neq q}^Q\mathbb{E}(|H_{E,q}|^2|H_{E,q'}|^2)\mathbb{E}\left(\mat{U} \mat{\Sigma}\mat{V}_1^H  \vect{e}_q \vect{e}_q^T \mat{V}_2 \mat{V}_2^H \vect{e}_{q'} \vect{e}_{q'}^T\mat{V}_1 \mat{\Sigma}^H   \mat{U}^H\right)\\
&=2\sum_{q=1}^Q\mathbb{E}\left(\mat{U} \mat{\Sigma}\mat{V}_1^H  \vect{e}_q \vect{e}_q^T \mat{V}_2 \mat{V}_2^H \vect{e}_{q} \vect{e}_{q}^T\mat{V}_1 \mat{\Sigma}^H   \mat{U}^H\right)\\
&+\sum_{q=1}^Q\sum_{q'\neq q}^Q\mathbb{E}\left(\mat{U} \mat{\Sigma}\mat{V}_1^H  \vect{e}_q \vect{e}_q^T \mat{V}_2 \mat{V}_2^H \vect{e}_{q'} \vect{e}_{q'}^T\mat{V}_1 \mat{\Sigma}^H   \mat{U}^H\right)\\
&=\sum_{q=1}^Q\mathbb{E}\left(\mat{U} \mat{\Sigma}\mat{V}_1^H  \vect{e}_q \vect{e}_q^T \mat{V}_2 \mat{V}_2^H \vect{e}_{q} \vect{e}_{q}^T\mat{V}_1 \mat{\Sigma}^H   \mat{U}^H\right)\\
&+\mathbb{E}\left(\mat{U} \mat{\Sigma}\mat{V}_1^H \sum_{q=1}^Q \vect{e}_q \vect{e}_q^T \mat{V}_2 \mat{V}_2^H \sum_{q'=1}^Q\vect{e}_{q'} \vect{e}_{q'}^T\mat{V}_1 \mat{\Sigma}^H   \mat{U}^H\right)\\
&=\sum_{q=1}^Q\mathbb{E}\left(\mat{U} \mat{\Sigma}\mat{V}_1^H  \vect{e}_q \vect{e}_q^T \mat{V}_2 \mat{V}_2^H \vect{e}_{q} \vect{e}_{q}^T\mat{V}_1 \mat{\Sigma}^H   \mat{U}^H\right)+\mathbb{E}\left(\mat{U} \mat{\Sigma}\mat{V}_1^H  \mat{V}_2 \mat{V}_2^H \mat{V}_1 \mat{\Sigma}^H   \mat{U}^H\right)
\end{align*}
Using the fact that $\mat{V}_2^H \mat{V}_1=\mat{0}$, the second term cancels and
\begin{align*}
\mathbb{E}\left(\vect{v}\vect{v}^H\right)&=\mathbb{E}\left(\mat{U} \mat{\Sigma}\mat{V}_1^H  \sum_{q=1}^Q\left(\vect{e}_q \vect{e}_q^T \mat{V}_2 \mat{V}_2^H \vect{e}_{q} \vect{e}_{q}^T\right)\mat{V}_1 \mat{\Sigma}^H   \mat{U}^H\right)
\end{align*}

If we assume (to be proven) that all elements of $\vect{v}$ have same variance, we can compute it as
\begin{align*}
\frac{1}{N}\mathbb{E}\left(\|\vect{v}\|^2\right)&=\frac{1}{N}\mathbb{E}\ \tr\left(\vect{v}\vect{v}^H\right)\\
&=\frac{1}{N}\mathbb{E} \ \tr\left( \mat{\Sigma}^2\mat{V}_1^H  \sum_{q=1}^Q\left(\vect{e}_q \vect{e}_q^T \mat{V}_2 \mat{V}_2^H \vect{e}_{q} \vect{e}_{q}^T\right)\mat{V}_1 \right)
\end{align*}
Let us rewrite $\mat{V}_1=\sum_{l}\vect{e}_{l}\vect{v}_{1,l}^H$ where $\vect{v}_{1,l}^H$ is the $l$-th row of $\mat{V}_1$ (of dimension $N\times 1$)
\begin{align*}
\frac{1}{N}\mathbb{E}\left(\|\vect{v}\|^2\right)&=\frac{1}{N}\sum_{q=1}^Q\sum_{l}\sum_{l'}\mathbb{E} \ \tr\left( \mat{\Sigma}^2\vect{v}_{1,l}\vect{e}_{l'}^T  \vect{e}_q \vect{e}_q^T \mat{V}_2 \mat{V}_2^H \vect{e}_{q} \vect{e}_{q}^T\vect{e}_{l}\vect{v}_{1,l}^H \right)\\
&=\frac{1}{N}\sum_{q=1}^Q\sum_{l}\sum_{l'}\delta_{l'-q}\delta_{l-q} \mathbb{E} \ \tr\left( \mat{\Sigma}^2\vect{v}_{1,l}\vect{e}_q^T \mat{V}_2 \mat{V}_2^H \vect{e}_{q} \vect{v}_{1,l}^H \right)\\
&=\frac{1}{N}\sum_{q=1}^Q \mathbb{E} \ \tr\left( \mat{\Sigma}^2\vect{v}_{1,q}\vect{e}_q^T \mat{V}_2 \mat{V}_2^H \vect{e}_{q} \vect{v}_{1,q}^H \right)
\end{align*}
Let us rewrite $\mat{V}_2=\sum_{l}\vect{e}_{l}\vect{v}_{2,l}^H$ where $\vect{v}_{2,l}^H$ is the $l$-th row of $\mat{V}_2$ (of dimension $Q-N\times 1$)
\begin{align*}
\frac{1}{N}\mathbb{E}\left(\|\vect{v}\|^2\right)&=\frac{1}{N}\sum_{q=1}^Q\sum_l\sum_{l'} \mathbb{E} \ \tr\left( \mat{\Sigma}^2\vect{v}_{1,q}\vect{e}_q^T \vect{e}_{l}\vect{v}_{2,l}^H \vect{v}_{2,l'} \vect{e}_{l'}^T \vect{e}_{q} \vect{v}_{1,q}^H \right)\\
&=\frac{1}{N}\sum_{q=1}^Q \mathbb{E} \ \tr\left( \mat{\Sigma}^2\vect{v}_{1,q}\vect{v}_{2,q}^H \vect{v}_{2,q} \vect{v}_{1,q}^H \right)\\
&=\frac{1}{N}\sum_{q=1}^Q \mathbb{E} \left( \| \vect{v}_{2,q}\|^2\vect{v}_{1,q}^H\mat{\Sigma}^2\vect{v}_{1,q}  \right)\\
&=\frac{1}{N}\sum_{q=1}^Q \mathbb{E} \left( \| \vect{v}_{2,q}\|^2 \| \vect{v}_{1,q}\|^2 \sum_n \sigma_n^2  \right)\\
\end{align*}
What I am not sure of (to be proven) is that $\sum_n \sigma_n^2$ will go to $N$, $\| \vect{v}_{1,q}\|^2$ will go to $ \frac{N}{Q} $ and $\| \vect{v}_{2,q}\|^2$ will go to $ \frac{Q-N}{Q}$ so that
\begin{align*}
\frac{1}{N}\mathbb{E}\left(\|\vect{v}\|^2\right)&=\frac{1}{N} Q   \frac{Q-N}{Q} \frac{N}{Q} N  \\
&=   \frac{Q-N}{Q} N   \\
\end{align*}

\textcolor{red}{Here, we should obtain, i.e., it converges to my simulation results:
\begin{align*}
\frac{1}{N}\mathbb{E}\left(\|\vect{v}\|^2\right)&= \frac{1}{Q} \left[ \frac{1}{N} Q   \frac{Q-N}{Q} \frac{N}{Q} N \right] \\
&=   \frac{Q-N}{Q^2} N   \\
&= \frac{U-1}{U^2} \\
\end{align*}
However, this is not equal to $\frac{1}{U+1}$, which is the real expression where my simulation converges. But, for high values of $U$, we have that $\frac{U-1}{U^2} \to \frac{1}{U+1}$. Typically, for $U=4$, we already observe a good match between the expressions.}
%However, this is not equal to $1/(U+1)=1/(N/Q+1)=Q/(N+Q)$!? (following what Sidney said)


\end{document}
