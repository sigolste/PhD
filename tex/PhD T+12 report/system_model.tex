\section{Communication Protocol}
\label{sec:communication-protocol}

A scheme of the secure \gls{fd} \gls{tr} \gls{siso} \gls{ofdm} communication is presented in Fig.\ref{fig:A_B_E_scheme} where Alice transmits wireless data to a legitimate receiver Bob. An eavesdropper (Eve) tries to eavesdrop the data. We assume that Alice does not have any information about Eve's \gls{csi} and perfectly knows Bob's instantaneous \gls{csi}. 
\begin{figure}[h!]
    \centering
    \includegraphics[width=.5\linewidth]{img/a_b_e_scheme.jpg}
    \caption{Security scenario}
    \label{fig:A_B_E_scheme}
\end{figure} 




%% Subsection
\subsection{Conventional FD TR SISO OFDM Communication}
The \gls{fd} \gls{tr} precoding scheme is illustrated in Fig.\ref{fig:TR_FD_classical}.  The communication is designed such that the data focuses at the legitimate receiver's position. 
\begin{figure}[h!]
    \centering
    \includegraphics[width=1\linewidth]{img/Capture.PNG}
    \caption{Conventional FD TR SISO OFDM system}
    \label{fig:TR_FD_classical}
\end{figure} 


The data is conveyed onto \gls{ofdm} symbols with $Q$ subcarriers. Without loss of generality, we consider that only one \gls{ofdm} block $\textbf{x}$ is sent over the \gls{fd} \gls{tr} precoding \gls{siso} \gls{ofdm} system. A data block $\textbf{x}$ is composed of $N$ symbols $x_n$ (for $n = 0,..., N-1$, with $N\leq Q$). The symbol $x_n$ is assumed to be a zero-mean \gls{rv} with variance $\EX{|x_n|^2} = \sigma_x^2 = 1$ (i.e., a normalized constellation is considered). The data block $\textbf{x}$ is then spread with a factor $U = Q/N$, called \gls{bor}, via the matrix $\spread$ of size $Q\times N$. The matrix $\spread$ is called the spreading matrix and stacks $U$ times $N\times N$ diagonal matrices, with diagonal elements taken from the set $\{\pm1\}$ and being \gls{iid} in order not to increase the \gls{papr} as suggested in \cite{4394231}. 
This matrix is normalized by a factor $\sqrt{U}$ in order to have $\spread^H \spread = \textbf{I}_N$:
\begin{equation}
\spread= \frac{1}{\sqrt{U}} \; . \;
   \begin{pmatrix}
    \pm 1 & 0 & \hdots & 0 \\
    0 & \pm 1 & \hdots & 0 \\
    \vdots & & \ddots & \vdots \\
    0 & 0 & \hdots & \pm 1 \\
     & \vdots& \vdots& \\
    \pm 1 & 0 & \hdots & 0 \\
    0 & \pm 1 & \hdots & 0 \\
    \vdots & & \ddots & \vdots \\
    0 & 0 & \hdots & \pm 1
 \end{pmatrix}
  \hspace{.2in} ; \hspace{.2in} [Q \times N]
 \label{eq:spread_mat}
\end{equation}
As stated in \cite{nguyen2019frequency}, the idea behind the spreading is that up-sampling a signal in the \gls{td} is equivalent to the repetition and shifting of its spectrum in the \gls{fd}. In doing so, each data symbol will be transmitted onto $U$ different subcarriers with a spacing of $N$ subcarriers, introducing frequency diversity. The spread sequence is then precoded before being transmitted. This requires the knowledge of Bob \gls{cfr} at Alice. We consider that Alice can perfectly estimate Bob \gls{cfr}. The channels between Alice and Bob ($\HB$) and between Alice and Eve ($\HE$) are assumed to be static during the transmission of one \gls{ofdm} symbol. $\HB$ and $\HE$ are $Q\times Q$ diagonal matrices whose elements are $h_{\text{B},q}$ and $h_{\text{E},q}$ (for $q = 0,...,Q-1$) and follow a zero-mean unit-variance complex normal distribution, i.e., their modulus follow a Rayleigh distribution. We also consider that the overall channel energies are normalized to unity for each channel realization. The precoding matrix $\HB^*$ is also a diagonal matrix with elements $h_{\text{B},q}^*$. At the receiver, a despreading operation is performed by applying $\spread^H$. We consider that Bob knows the spreading sequence and apply a \gls{zf} equalization.  In the following, different decoding structures $\textbf{G}$ will be investigated at Eve. These different schemes will lead to different level of security performances. A perfect synchronization is also assumed at Bob and Eve positions.\\

An illustration of such a communication is presented in Fig.\ref{fig:no_AN_illustration} where a block of $N=4$ symbols is spread by a factor $U=4$ and then sent via $Q=16$ subcarriers. We observe that, at Bob, the received sequence is perfectly recovered which is not the case at Eve's position. 
\begin{figure}[h!]
    \centering
    \includegraphics[width=1\linewidth]{img/scheme_no_AN_illustration.png}
    \caption{Illustration of conventional FD TR SISO OFDM system}
    \label{fig:no_AN_illustration}
\end{figure} 






\subsubsection{Received sequence at the intended position}
After despreading, the received sequence at Bob is:
\begin{equation}
    \textbf{y}_{\text{B}}^H = \spread^H  \module{\HB}^2\spread\; \textbf{x} +  \spread^H \textbf{v}_\text{B} 
    \label{eq:rx_bob}
\end{equation}
where $\textbf{v}_\text{B}$ is the \gls{fd} complex \gls{awgn}. The noise's auto-correlation is $\EX{|v_{\text{B},n}|^2}  = \sigma_{\text{V,B}}^2$ and the covariance matrix is $\EX{(\spread^H  \textbf{v}_\text{B}) . (\spread^H \textbf{v}_\text{B})^H} = \sigma_{\text{V,B}}^2 . \textbf{I}_N$. We also assume that the data symbol $x_n$ and noise $v_{\text{B,n}}$ are independent of each other. In (\ref{eq:rx_bob}), each transmitted symbol is affected by a real gain at the position of the legitimate receiver since the product $\HB\;\HB^*$ is a real diagonal matrix. The gains differ between each symbol in the \gls{ofdm} block but increases with an increase of the \gls{bor} value as each symbol would be sent on more subcarriers and would benefit from a larger frequency diversity gain. If we consider a fixed bandwidth, the \gls{tr} focusing effect is enhanced for higher \gls{bor}'s at the expense of the data rate. After \gls{zf} equalization, we obtain:
\begin{equation}
    \hat{\textbf{x}}_{\text{B}} = \left(\spread^H \module{\HB}^2\spread\right)^{-1} \left(\spread^H  \module{\HB}^2\spread\; \textbf{x} +  \spread^H \textbf{v}_\text{B}\right) = \textbf{x} + \left(\spread^H  \module{\HB}^2 \spread^H \right)^{-1} \spread^H \textbf{v}_\text{B}
    \label{eq:rx_bob_eq}
\end{equation}
From (\ref{eq:rx_bob_eq}), we observe that the transmit data is perfectly recovered at high \gls{snr}.




\subsubsection{Received sequence at the unintended position}
The data received at the unintended position is given by:
\begin{equation}
    \textbf{y}_{\text{E}}^G= \textbf{G}\HE \HB^* \spread \textbf{x}  +  \textbf{v}_\text{E}
    \label{eq:rx_eve}
\end{equation}
where $\textbf{G}$ is a $N \times N$ filter matrix performed by Eve, $\textbf{v}_\text{E}$ is the complex \gls{awgn}. The noise auto-correlation is $\EX{|v_{\text{E,n}}|^2} = \sigma_{\text{V,E}}^2$ and the covariance matrix is $\EX{(\spread^H  \textbf{v}_\text{E}) . (\spread^H \textbf{v}_\text{E})^H} = \sigma_{\text{V,E}}^2 . \textbf{I}_N$.  In (\ref{eq:rx_eve}), $\HE \HB^*$ is a complex diagonal matrix. Therefore, due to the precoding, i.e., since the data transmission is designed to reach Bob position, each received symbol component will be affected by a random complex coefficient. The magnitude of this coefficient does not depend on the \gls{bor} value. It results in an absence of \gls{tr} gain at the unintended position. As a consequence, worse decoding performance is obtained compared to the intended position. Eve needs lower noise power than Bob to reach the same \gls{ber}. After \gls{zf} equalization, one obtains:
\begin{equation}\
    \hat{\textbf{x}}_{\text{E}} = \left(\textbf{G}  \HE \HB^*\spread\right)^{-1} \left(\textbf{G}\HE \HB^* \spread \textbf{x}  +  \textbf{v}_\text{E}\right) = \textbf{x} +  \left(\textbf{G}  \HE \HB^*\spread\right)^{-1}\textbf{G}\textbf{v}_\text{E}
    \label{eq:rx_eve_eq}
\end{equation}
Equation (\ref{eq:rx_eve_eq}) shows that in the classical \gls{fd} \gls{tr} \gls{siso} \gls{ofdm} communication scheme, the data could potentially be recovered at Eve's position. A similar \gls{ber} could be obtained at Eve if she is closer to Alice than Bob is and/or if its noise power is less than Bob's one. This motivates the addition of \gls{an} in order to corrupt the data detection at any unintended positions in order to secure the communication. In Section \ref{sec:perf}, different filtering structures $\textbf{G}$ will be investigated leading to different security performances of the scheme.

%% Subsection
\subsection{FD TR SISO OFDM communication with AN addition}

\begin{figure}[htb!]
    \centering
    \includegraphics[width=1\linewidth]{img/com_scheme_an.PNG}
    \caption{FD TR SISO OFDM system with added artificial noise}
    \label{fig:TR_FD_AN}
\end{figure} 
In order to secure the communication between Alice and Bob, an \gls{an} signal $\w$ is added after precoding to the useful signal $\textbf{x}_S$ at the transmitter side, as depicted in Fig. \ref{fig:TR_FD_AN}. The \gls{an} should not have any impact at Bob's position but should be seen as interference everywhere else since Alice does not have any information about Eve's \gls{csi}. Furthermore, this signal should not be guessed at the unintended positions to ensure the secure communication. With these considerations, the transmitted sequence becomes:
\begin{equation}
    \textbf{x}_{\text{TR}} = \sqrt{\alpha} \;\HB^*  \spread\; \textbf{x} +  \sqrt{1-\alpha} \; \w
    \label{eq:sym_rad_AN}
\end{equation} 
where $\alpha \in [0,1]$ defines the ratio of the total power dedicated to the useful signal, knowing that $\EX{\module{\HB^*\spread\textbf{x}}^2} = \EX{\module{\w}^2} = 1/U$. Whatever the value of $\alpha$, the total transmitted power remains constant.\\

Fig.\ref{fig:AN_illustration} illustrates a scheme with additive \gls{an}. We sent a block of $N=4$ symbols which is spread by a factor $U=4$, i.e., the data is conveyed onto $Q=16$ subcarriers. A block of 16 \gls{an} symbols is added (in pink). At bob's position, there is no influence of these \gls{an} symbols and the data is perfectly recovered. At Eve, the received symbols are corrupted by the \gls{an} signal and by the data precoding.
\begin{figure}[h!]
    \centering
    \includegraphics[width=1\linewidth]{img/scheme_AN_illustration.png}
    \caption{Illustration of FD TR SISO OFDM system with added artificial noise}
    \label{fig:AN_illustration}
\end{figure} 

 


\subsubsection{AN Design}
In order not to have any impact at the intended position, the AN signal must satisfy the following condition:
\begin{equation}
    \textbf{A} \w \; = \; \textbf{0}
    \label{eq:an_cond}
\end{equation}
where $\textbf{A} = \spread^H\HB\; \in \C^{N\times Q}$. Condition (\ref{eq:an_cond}) ensures that $\w$ lies in the right null space of $\textbf{A}$. If we perform a \gls{svd} of $\textbf{A}$, we obtain:
\begin{equation}
    \textbf{A} = \textbf{U} 
    \begin{pmatrix}
    \Sigma \; \textbf{0}_{Q-N\times Q}
    \end{pmatrix}
    \begin{pmatrix}
    \textbf{V}_1^H \\
    \textbf{V}_2^H
    \end{pmatrix}
    \label{eq:an_svd}
\end{equation}
where $\textbf{U} \in \C^{N \times N}$ contains left singular vectors, $\Sigma \in \C^{N \times N}$ is a diagonal matrix containing singular values, $\textbf{V}_1 \in \C^{Q \times N}$ contains right singular vectors associated to non-zero singular values, and $\textbf{V}_2 \in \C^{Q \times Q-N}$ contains right singular vectors that span the right null space of $\textbf{A}$. Therefore, the \gls{an} signal can be expressed as:
\begin{equation}
    \w = \beta \textbf{V}_2 \tilde{\w}
    \label{eq:an_w}
\end{equation}
which ensures that (\ref{eq:an_cond}) is satisfied for any arbitrary vector $\tilde{\w} \in \C^{Q-N \times 1}$. Since $Q = NU$, as soon as $U\geq 2$, there is a set of infinite possibilities to generate $\tilde{\w}$ and therefore the \gls{an} signal. In the following, we assume that $\tilde{\w}$ is a zero-mean circularly symmetric white complex Gaussian noise with covariance matrix $\EX{\tilde{\w}(\tilde{\w})^H} = \textbf{I}_{Q-N \times 1}$. The \gls{an} signal is then generated thanks to (\ref{eq:an_w}) and finally weighted by $\beta$ to have an energy of 1/U.


\subsubsection{Received sequence at the intended position}
After despreading, the received sequence at Bob is: 
\begin{equation}
    \textbf{y}_{\text{B}}^H = \sqrt{\alpha} \; \spread^H \module{\HB}^2 \spread \textbf{x} \;  +  \;  \spread^H \textbf{v}_\text{B} 
    \label{eq:rx_bob_AN}
\end{equation}
Again, each transmitted symbol is affected by a real gain depending on the BOR value and weighted by $\sqrt{\alpha}$. One can observe that no \gls{an} contribution is present in (\ref{eq:rx_bob_AN}) since (\ref{eq:an_cond}) are respected. A \gls{zf} equalization is performed at the receiver leading to:
\begin{equation}
    \hat{\textbf{x}}_{\text{B}} = \left( \sqrt{\alpha} \spread^H \module{\HB}^2 \spread \right)^{-1}  \left(\sqrt{\alpha}  \spread^H\module{\HB}^2 \spread \textbf{x}   +    \spread^H \textbf{v}_\text{B}\right)= \textbf{x} + \left( \sqrt{\alpha} \spread^H \module{\HB}^2 \spread \right)^{-1} \spread^H \textbf{v}_\text{B}
    \label{eq:rx_bob_AN_eq}
\end{equation}
From (\ref{eq:rx_bob_AN_eq}), a perfect data recovery is possible in high \gls{snr} scenario.


\subsubsection{Received sequence at the unintended position}
The received sequence at the eavesdropper position has the form:
\begin{equation}
    \textbf{y}_{\text{E}}^G = \sqrt{\alpha}  \textbf{G} \HE \HB^* \spread\textbf{x} + \sqrt{1-\alpha} \textbf{G} \HE \w + \textbf{G}  \textbf{v}_\text{E}
    \label{eq:rx_eve_an}
\end{equation}
In (\ref{eq:rx_eve_an}), a term depending on the \gls{an} signal appears since $\textbf{G}\HE \w \neq \textbf{0}$. This term introduces an interference at Eve and thus scrambles the received constellation even in a noiseless environment. After \gls{zf} equalization, the estimated symbols are:
\begin{equation}
    \begin{split}
         \hat{\textbf{x}}_{\text{E}} =& \left(\textbf{G} \HE \HB^* \spread \right)^{-1}
         \left( \sqrt{\alpha} \textbf{G} \HE \HB^* \spread \textbf{x} +   \sqrt{1-\alpha} \textbf{G} \HE \w  +  \textbf{G}  \textbf{v}_\text{E}  \right) \\
         =& \;\sqrt{\alpha}\textbf{x} + \sqrt{1-\alpha} \left(\textbf{G} \HE \HB^* \spread \right)^{-1}  \textbf{G} \HE \w + \left(\textbf{G} \HE \HB^* \spread \right)^{-1}  \textbf{G} \textbf{v}_\text{E}
    \end{split}
    \label{eq:rx_an_eve_eq}
\end{equation}
Equation (\ref{eq:rx_an_eve_eq}) shows that the addition of \gls{an} in the \gls{fd} \gls{tr} \gls{siso} \gls{ofdm} communication can secure the data transmission. It is to be noted that, since $\w$ is generated from an infinite set of possibilities, even if Eve knows its equivalent channel $\HE\HB^*$ and the spreading sequence, she cannot estimate the \gls{an} signal  to try retrieving the data.  The degree of security will depend on the amount of \gls{an} energy that is injected into the communication and on the decoding capabilities of Eve, as shown in Section \ref{sec:perf}.


