\renewcommand{\theparagraph}{C}

\section{Momentum computation of complex-valued random normal variables}
\label{appC}
\subsection{Real-valued random normal variable}
For real-valued random variable, the moment-generating function is an alternative specification of its probability distribution. In particular, it allows to compute the moments of the probability distribution as:
\begin{equation}
m_n = \EX{X^n} = M_X^{(n)}(0) = \left. \frac{d^n M_X}{dt^n}\right|_{t=0}
\end{equation}
For a real normal random variable $\mathcal{N}(\mu,\sigma^2) $, the moment-generating function is given by:
\begin{equation}
M_X = e^{t \mu + \frac{1}{2} \sigma^2 t^2 }
\end{equation}
From that, we have:
\begin{equation}
\begin{split}
&\EX{|X|^2} = \sigma^2 + \mu^2\\
&\EX{|X|^4} = 3(\sigma^2)^2 + 6 \sigma^2 \mu^2 + \mu^4
\end{split}
\label{eq:moments_x}
\end{equation}

\subsection{Complex-valued random normal variable}
A complex normal random variable is defined as $Z = X + iY$ where $X \sim \mathcal{N}(\mu_x,\sigma^2_x)$ and $Y \sim \mathcal{N}(\mu_y,\sigma^2_y)$. We also have:
\begin{equation}
\begin{split}
&|Z|^2 = X^2 + Y^2\\
&|Z|^4 = X^4  + 2 X^2 Y^2 + Y^4
\end{split}
\label{eq:deriv_z}
\end{equation}
Since $X$ and $Y$ are independent and taking into account eq.\ref{eq:moments_x}, the moments of $Z$ are easy to compute:
\begin{equation}
\begin{split}
&\EX{|Z|^2} = \sigma_x^2 + \mu_x^2 + \sigma_y^2 + \mu_y^2\\
&\EX{|Z|^4} = 3(\sigma_x^2)^2 + 6 \sigma_x^2 \mu_x^2 + \mu_x^4 + 2\left[ ( \sigma_x^2 + \mu_x^2 )(\sigma_y^2 + \mu_y^2)\right] + 3(\sigma_y^2)^2 + 6 \sigma_y^2 \mu_y^2 + \mu_y^4
\end{split}
\label{eq:moments_z}
\end{equation}