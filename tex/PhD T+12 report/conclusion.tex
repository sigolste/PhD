\section{Conclusion \& Perspectives}
\label{sec:ccl}
The problem of securing the \gls{fd} \gls{tr} \gls{siso} \gls{ofdm} wireless transmission from a transmitter to a legitimate receiver in the presence of a passive eavesdropper is considered. A novel and original approach based on the addition of an \gls{an} signal onto \gls{ofdm} blocks that improves the \gls{pls} is proposed. This approach can be easily integrated into existing standards based on \gls{ofdm}. It only requires a single transmit antenna and is therefore well suited for devices with limited capabilities. Analytic and simulation results show that the novel approach significantly improves the security of the communication and so considerably jeopardizes any attempt of an eavesdropper to retrieve the data. \\

In this work, four eavesdropper decoding structures are investigated giving rise to different secrecy performances. From the previous discussion, we remark that a simple matched filtering decoding structure strongly impacts the secrecy rate of the communication. As a result, one of the aspects of the future work will be to implement a more robust scheme. New \gls{an} injection methods and/or new precoding techniques will be investigated to try to further improve the security of the communication. \\

So far, only a \gls{siso} system has been considered. A natural extension of the work will be to consider a \gls{mimo} system. That way, in addition to benefiting from the  frequency diversity of the channels, we will benefit from the spatial diversity. In doing so, any attempt to try to eavesdrop the data will be expected to become harder. In addition, we will consider that the passive eavesdropper will be equipped with multiple antennas which will gives him more decoding capabilities. \\

Furthermore, only a single-user communication is now considered. A multi-users scheme can be designed where each user channel can contribute to secure the whole communication. As a consequence, the level of security of the scheme is expected to improve.\\

At this point of the work, a very simple channel model was implemented. In fact, a multi-taps channel where each tap is Rayleigh distributed is considered. No correlation between the different subcarriers, i.e. frequency correlation, or spatial correlation between Bob and Eve channels have been taken into account. This leads to results that are relatively easy to compare with analytic models but that do not represent well real life scenarios. A natural extension of the work will be to consider these correlations in order to obtain more realistic results. \\

Finally, the objective will be to test the \gls{fd} \gls{tr} \gls{ofdm} system with the \gls{usrp} devices in real indoor environments.
\label{sec:ccl}

