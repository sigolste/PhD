\section{Introduction}

Due to their broadcast nature, wireless communications remain unsecured. With the  deployment of 5G as an heterogeneous network possibly involving different radio access technologies, \gls{pls} has gained recent interests in order to secure wireless communications, \cite{alves2012performance,yang2012physical,tran2015secrecy}. \gls{pls} classically takes benefit of the characteristics of wireless channels, such as multipath fading, to improve security of communications against potential eavesdroppers. A secure communication can exist as soon as the eavesdropper channel is degraded with respect to the legitimate user one, \cite{wyner1975wire}. This can be achieved by increasing the \gls{sinr} at the intended position and decreasing the \gls{sinr} at the unintended position if its \gls{csi} is known, and/or, by adding an \gls{an} signal that lies in the null space of the legitimate receiver's channel. While many works implement these schemes using multiple antennas at the transmitter, only few ones intend to do so with \gls{siso} systems \cite{li2013waveform,xu2018security,li2018artificial,li2017artificial}.

In \cite{li2013waveform}, a technique is proposed that combines a symbol waveform optimisation in \gls{td} to reach a desired \gls{sinr} at the legitimate receiver and an \gls{an} injection using the remaining available power at the transmitter when eavesdropper's \gls{csi} is not known. Another approach to increase the \gls{sinr} in \gls{siso} systems is \gls{tr}. This has the advantage to be implemented with a simple precoder at the transmitter. \gls{tr} achieves a gain at the intended receiver position only, thereby naturally offering a possibility of secure communication, \cite{oestges2005characterization}. \gls{tr} is achieved by up/downsampling the signal in the TD. It as been shown in  \cite{nguyen2019frequency} that \gls{tr} can be equivalently achieved in \gls{fd} by replicating and shifting the signal spectrum. \gls{fd} implementation has the advantage to be easily performed using \gls{ofdm}. To further enhance the secrecy, few works combine TD \gls{tr} precoding with \gls{an} injection \cite{xu2018security,li2018artificial,li2017artificial}. In these works, the \gls{an} is added either on all the channel taps or on a set of selected taps. While the condition for \gls{an} generation is given, its derivation is however not detailed. Furthermore, the impact of \gls{bor}, defined as the up/downsampling rate \cite{dubois2010use}, has not been yet studied in the literature. 

An approach to establish secure communication using a \gls{fd} \gls{tr} precoder in \gls{siso} \gls{ofdm} systems is proposed. An \gls{an} signal is designed to maximize the secrecy rate (SR) of the communication in presence of a passive eavesdropper whose \gls{csi} is supposed unknown. The proposed scheme uses only frequency diversity inherently present in multipath environments to achieve security. It can therefore be used in \gls{siso} systems and is then well-suited for resource-limited nodes such as encountered in \gls{iot} for instance.  Indeed, \gls{mimo} capabilities require several antennas and as many transceivers and ADC/DAC, which might not fit into small-size sensors and could  be too power-consuming for such IoT scenarios. Furthermore, the \gls{ofdm} implementation makes this approach compatible with LTE and 5G systems. \\

\textit{Notation:} the italic lower-case letter denotes a complex number. Greek letter corresponds to a scalar, the bold lower-case letter denotes a column vector. Bold upper-case letter corresponds to a matrix; $\textbf{I}_N$ is $N \times N$ identity matrix; $(.)^{-1}$, $(.)^{*}$, $(.)^{H}$ are respectively the inverse, the complex conjugate and the Hermitian transpose operators; $\EX{.}$ is the expectation operator; $\module{.}$ is the modulus operator (element-wize modulus if we deal with a matrix); $\odot$ is the element-wize (hadamard) product between two vectors of same dimension.