
\documentclass[journal,comsoc]{IEEEtran}
\usepackage[T1]{fontenc}% optional T1 font encoding




% *** CITATION PACKAGES ***
\usepackage{cite}
\usepackage{color}
% *** GRAPHICS RELATED PACKAGES ***
%
\ifCLASSINFOpdf
   \usepackage[pdftex]{graphicx}

\else
  
  \usepackage[dvips]{graphicx}
  
\fi

% *** MATH PACKAGES ***
%
\usepackage{amsmath}
\interdisplaylinepenalty=2500
\usepackage[cmintegrals]{newtxmath}
\usepackage{bm}

% *** SPECIALIZED LIST PACKAGES ***
\usepackage{algorithmic}

% *** ALIGNMENT PACKAGES ***
\usepackage{array}

% *** LOREM LIPSUM PACKAGE ***
\usepackage{lipsum}  

% *** SUBFIGURE PACKAGES ***
\ifCLASSOPTIONcompsoc
  \usepackage[caption=false,font=normalsize,labelfont=sf,textfont=sf]{subfig}
\else
  \usepackage[caption=false,font=footnotesize]{subfig}
\fi

% *** FLOAT PACKAGES ***
\usepackage{fixltx2e}
\usepackage{dblfloatfix}

\ifCLASSOPTIONcaptionsoff
  \usepackage[nomarkers]{endfloat}
 \let\MYoriglatexcaption\caption
 \renewcommand{\caption}[2][\relax]{\MYoriglatexcaption[#2]{#2}}
\fi

% *** PDF, URL AND HYPERLINK PACKAGES ***
\usepackage{url}

% correct bad hyphenation here
\hyphenation{op-tical net-works semi-conduc-tor}






%%%%%%%%%%%%%%%%%%%%%%%%%%%%%%%%%%%%%%%%%%%%%%%%%%%%%%%







\begin{document}
\title{Still to define}
\author{\IEEEauthorblockN{Sidney Golstein\IEEEauthorrefmark{1}\IEEEauthorrefmark{2},
		Trung-Hien Nguyen\IEEEauthorrefmark{1},
		Fran\c cois Rottenberg\IEEEauthorrefmark{1},
		Fran\c cois Horlin\IEEEauthorrefmark{1}, 
		Philippe De Doncker\IEEEauthorrefmark{1}, and
		Julien Sarrazin\IEEEauthorrefmark{2}} \\
\IEEEauthorblockA{\IEEEauthorrefmark{1}Wireless Communication Group,
	Universit\'{e}  Libre de Bruxelles, 1050 Brussels, Belgium} \\
\IEEEauthorblockA{\IEEEauthorrefmark{2}Sorbonne Université, CNRS, Laboratoire de Génie Electrique et Electronique de Paris, 75252, Paris, France \\
	Université Paris-Saclay, CentraleSupélec, CNRS, Laboratoire de Génie Electrique et Electronique de Paris, 91192, Gif-sur-Yvette, France} \\
\IEEEauthorblockA{\small{\{sigolste,trung-hien,francois.rottenberg,fhorlin,philippe.dedoncker\}@ulb.ac.be}} \\
\IEEEauthorblockA{\small{julien.sarrazin@sorbonne-universite.fr}}
\thanks{This work was supported by the ANR GEOHYPE project, grant ANR-16-CE25-0003 of the French Agence Nationale de la Recherche and was also carried out in the framework of COST Action CA15104 IRACON.}}

% The paper headers
\markboth{Journal of \LaTeX\ Class Files,~Vol.~14, No.~8, August~2015}%
{Shell \MakeLowercase{\textit{et al.}}: Bare Demo of IEEEtran.cls for IEEE Communications Society Journals}


% make the title area
\maketitle

% As a general rule, do not put math, special symbols or citations
% in the abstract or keywords.
\begin{abstract}
The abstract goes here.
\end{abstract}

% Note that keywords are not normally used for peerreview papers.
\begin{IEEEkeywords}
Communications Society, IEEE, IEEEtran, journal, \LaTeX, paper, template.
\end{IEEEkeywords}

\IEEEpeerreviewmaketitle





%%%%%%%%%   INTRODUCTION   %%%%%%%%% 
\section{Introduction}
\IEEEPARstart{T}{his} demo file is intended to serve as a ``starter file''
for IEEE Communications Society journal papers produced under \LaTeX\ using
IEEEtran.cls version 1.8b and later.
% You must have at least 2 lines in the paragraph with the drop letter
% (should never be an issue)
I wish you the best of success.







%%%%%%%%%   SYSTEM MODEL   %%%%%%%%% 
\section{System Model}

\subsection{Establishment Protocol}
{\color{red}Parler des 3 differents protocol de communication permettant d'etablir la comm avec les 3 schemes}



\subsection{Communication Protocol}
{\color{red}Schema de communication avec addition d AN}

\subsubsection{Received sequence at the intended }
\subsubsection{Received sequence at the unintended}

\subsection{Artificial noise Design}



%%%%%%%%%   PERFORMANCE   %%%%%%%%% 
\section{Performance Assessments}


\subsection{Hypothesis}
{\color{red} Mettre les hypotheses p7 du rapport}

\subsection{SINR determination}
\subsubsection{At the intended position}

%%
\subsubsection{At the unintended position}
\paragraph{Same decoding strucure as Bob}
\paragraph{Matched filtering}
\paragraph{Own channel knowledge}

%%
\subsection{Optimal amount of AN energy to inject}
\subsubsection{Same decoding strucure as Bob}
\subsubsection{Matched filtering}
\subsubsection{Own channel knowledge}

%%
\subsection{Secrecy rate optimization via waterfilling}




%%%%%%%%%   SIMULATION RESULTS   %%%%%%%%% 
\section{Simulation Results}

\subsection{Comparaison between the different models}

\subsection{Comparaison between models and simulations}

\subsection{Waterfilling optimization}

%%%%%%%%%   CONCLUSONS   %%%%%%%%% 
\section{Conclusions}


\subsection{Subsection Heading Here}
\lipsum[1]

\subsubsection{Subsubsection Heading Here}

\lipsum[1]



%\begin{figure}[!t]
%\centering
%\includegraphics[width=2.5in]{myfigure}
% where an .eps filename suffix will be assumed under latex, 
% and a .pdf suffix will be assumed for pdflatex; or what has been declared
% via \DeclareGraphicsExtensions.
%\caption{Simulation results for the network.}
%\label{fig_sim}
%\end{figure}



% An example of a double column floating figure using two subfigures.

%\begin{figure*}[!t]
%\centering
%\subfloat[Case I]{\includegraphics[width=2.5in]{box}%
%\label{fig_first_case}}
%\hfil
%\subfloat[Case II]{\includegraphics[width=2.5in]{box}%
%\label{fig_second_case}}
%\caption{Simulation results for the network.}
%\label{fig_sim}
%\end{figure*}



%\begin{table}[!t]
% increase table row spacing, adjust to taste
%\renewcommand{\arraystretch}{1.3}
%\caption{An Example of a Table}
%\label{table_example}
%\centering
% Some packages, such as MDW tools, offer better commands for making tables
% than the plain LaTeX2e tabular which is used here.
%\begin{tabular}{|c||c|}
%\hline
%One & Two\\
%\hline
%Three & Four\\
%\hline
%\end{tabular}
%\end{table}









%%%%%%%%%   APPENDICES  %%%%%%%%% 
\appendices
\section{Proof of the First Zonklar Equation}
Appendix one text goes here.

\section{Okay}
Appendix two text goes here.


% use section* for acknowledgment
\section*{Acknowledgment}


The authors would like to thank...

\ifCLASSOPTIONcaptionsoff
  \newpage
\fi

%%%%%%%%%   BIBLIOGRAPHY   %%%%%%%%% 
\begin{thebibliography}{4}

\bibitem{IEEEhowto:kopka}
H.~Kopka and P.~W. Daly, \emph{A Guide to \LaTeX}, 3rd~ed.\hskip 1em plus
  0.5em minus 0.4em\relax Harlow, England: Addison-Wesley, 1999.

\end{thebibliography}


%%%%%%%%%   BIOGRAPHY   %%%%%%%%% 
\begin{IEEEbiography}{Michael Shell}
Biography text here.
\end{IEEEbiography}

% if you will not have a photo at all:
\begin{IEEEbiographynophoto}{John Doe}
Biography text here.
\end{IEEEbiographynophoto}


\begin{IEEEbiographynophoto}{Jane Doe}
Biography text here.
\end{IEEEbiographynophoto}


\end{document}


