%%%%%%%%%%%%%%%%
%%% Packages %%%
%%%%%%%%%%%%%%%%
%%%%%%%%%%%%%%%
%%% Général %%%
%%%%%%%%%%%%%%%
\renewcommand{\baselinestretch}{1.2} 
\usepackage[utf8]{inputenc}
\usepackage[english]{babel}
\usepackage[T1]{fontenc}
\usepackage{mathpazo}
\usepackage{lmodern}
\usepackage{courier}

% \usepackage{graphics} % DEPRECATED
\usepackage{graphicx}
\usepackage{longtable}
% \usepackage{subfigure} % DEPRECATED
\usepackage{subcaption}

%%%%%%%%%%
% Autres %
%%%%%%%%%%
%\usepackage{layout}
%\usepackage{geometry}
%\usepackage{setspace}

\usepackage[normalem]{ulem}
%\usepackage{eurosym}
%\usepackage{bookman}
%\usepackage{charter}
%\usepackage{newcent}
%\usepackage{lmodern}
%\usepackage{mathpazo}
%\usepackage{mathptmx}
%\usepackage{moreverb}
%\usepackage{fancyhdr}
%\usepackage{color}
\usepackage{tikz}
\usetikzlibrary{shapes,arrows}
\usepackage{nomencl}
\usepackage{makeidx}
\usepackage{fullpage}
\usepackage{verbatim}
\usepackage{titlesec}
\usepackage[english]{minitoc}
\usepackage[hyphens]{url}       %pour créer des url
\usepackage{hyperref}
\usepackage{multicol}
\usepackage{lscape}
\usepackage{supertabular}
\usepackage{wrapfig}
%\usepackage{rotating}          %rotation d'une page
%\usepackage{eurosym}            %symbole euro
\usepackage{colortbl}
%\usepackage[table]{xcolor}
\usepackage{cancel}



\usepackage[font={footnotesize}]{caption}

\usepackage[nottoc, notlof, notlot]{tocbibind}
\usepackage{microtype}
\usepackage{epstopdf}
\usepackage[scaled]{beramono}
\usepackage{gensymb}

\usepackage{pbox}

\usepackage[final]{pdfpages} %pour insérer des pdf dans le latex




\setcounter{secnumdepth}{4}


%% ----------------------------------------------------------------
% package titres
%% ----------------------------------------------------------------
\usepackage{titlesec}
%%%% Format titre  %%%%
\titleformat{\chapter}[display]{\sf\LARGE}
{\vspace{-01ex}
\filleft{\chaptertitlename}~\Huge\thechapter}
{4ex}{\titlerule\vspace{2ex}\filright}[\vspace{2ex}\titlerule]



\titleformat{\paragraph}
{\normalfont\normalsize\bfseries}{\theparagraph}{1em}{}
\titlespacing*{\paragraph}
{0pt}{3.25ex plus 1ex minus .2ex}{1.5ex plus .2ex}


\usepackage{tikz}
\usetikzlibrary{shapes,arrows}

%%%% Format titre  %%%%

%\titleformat{\chapter}[display]{\sf\bfseries\LARGE}
%{\vspace{-10ex}
%\filleft\MakeUppercase{\chaptertitlename}~\Huge\thechapter}
%{4ex}{\titlerule\vspace{2ex}\filright}[\vspace{2ex}\titlerule]
%{\vspace{-15ex}}





% Table of contents %

%\setcounter{tocdepth}{3}            %niveau jusqu'où ça s'affiche dans la tdm
%\setcounter{secnumdepth}{4}         %niveau pour afficher les numéros
%\renewcommand{\thesubsubsection}{\roman{subsubsection}}
%\renewcommand{\thechapter}{\Roman{chapter}} 
%\renewcommand{\thesubsubsection}{\alph{subsubsection})}         %subsubsection numérotées avec des lettres


%%% Mise en forme du code Matlab %%%

\usepackage{listings}
\usepackage{listingsutf8}

\definecolor{orange}{HTML}{FFCC66}
\definecolor{mygreen}{RGB}{0,153,51}
\definecolor{mylilas}{RGB}{170,55,241}
\definecolor{mygrey}{RGB}{160,160,160}




\usepackage{amsmath,amssymb}

\DeclareRobustCommand{\bbone}{\text{\usefont{U}{bbold}{m}{n}1}}







\lstset{
    frame=single,
    breaklines=true,
    postbreak=\raisebox{0ex}[0ex][0ex]{\ensuremath{\color{mygrey}\hookrightarrow\space}},
    %inputencoding=utf8/latin1,
}

\lstset{language=Matlab,%
    %basicstyle=\color{red},
    breaklines=true,
    keywordstyle=\color{blue},
    morekeywords=[2]{1}, keywordstyle=[2]{\color{black}},
    identifierstyle=\color{black},
    stringstyle=\color{mylilas},
    commentstyle=\color{mygreen},
    showstringspaces=false,%without this there will be a symbol in the places where there is a space
    numbers=left,%
    numberstyle={\tiny \color{black}},% size of the numbers
    numbersep=9pt, % this defines how far the numbers are from the text
    emph=[1]{for,end,break},emphstyle=[1]\color{red}, %some words to emphasise
}



%%% Tableau %%%
\usepackage{tabularx} %Permet d'auto dimensionner les tableaux




%%% Graphiques %%%
\usepackage{tikz}
\usepackage{pgfplots}
\usepackage{circuitikz}
\usepackage{sidecap}
\usepackage[font=footnotesize]{caption}


%%% Mise en page %%%
\usepackage{amsmath}
\usepackage{amsfonts}
\usepackage{amssymb}
\usepackage{amsthm}
\usepackage[tt]{titlepic}% Centre le titre
\usepackage{fancyhdr}   % Permet de modifier l'entête & footer
%\usepackage{caption}     % Permet d'ajouter des légendes en images sans les mettre en float + dans la marge
\usepackage{wrapfig}
\usepackage{fullpage}
\usepackage{multicol}   % pour les liste sur plusieurs colonnes
\usepackage{float}      %permet de mettre du texte entre les figures grace a [H]. Génial! 
\usepackage{eso-pic}    % Fond d'écran page de garde
\usepackage{adjustbox}  % Empêche les box de sortir de la page
\usepackage[subnum]{cases}
\usepackage[margin=.75in]{geometry}



%%% Math %%%
\usepackage{delarray} % Belles matrices
\usepackage{siunitx}
%\sisetup{locale = FR,detect-all}
% Pour mettre siunitx en mode français (virgule plutôt que point etc.)

%%% Codes %%%
\usepackage{listings}

%% Reference
\usepackage{hyperref}
\renewcommand*{\figureautorefname}{fig.}
\def\appendixautorefname{annexe}
\def\tableautorefname{tab.}
\renewcommand*{\chapterautorefname}{ch.}



%%%%%%%%%%%%%%%%%
%%% Commandes %%%
%%%%%%%%%%%%%%%%%

%%% Physique %%%
\newcommand{\cst}{\text{cst}}
\newcommand{\E}{\vec E}
\newcommand{\B}{\vec B}
\newcommand{\F}{\vec F}
\newcommand{\module}[1]{\left|#1\right|}
\newcommand{\EX}[1]{\mathbb{E} \left[#1\right]}% expected value
\newcommand{\HE}{\textbf{H}_{\text{E}}}
\newcommand{\HB}{\textbf{H}_{\text{B}}}
\newcommand{\vb}{\textbf{v}_{\text{B}}}
\newcommand{\ve}{\textbf{v}_{\text{E}}}
\newcommand{\spread}{\textbf{S}}
\newcommand{\SH}{\textbf{S}^H}
\newcommand{\w}{\textbf{w}}

%%% Math %%%
\newcommand{\oiint}{\int\!\!\!\!\!\!\! \:\!\subset\!\!\supset\!\!\!\!\!\!\!\int}
\newcommand{\rot}{\text{rot}\,}
\newcommand{\divv}{\text{div}\,}
\newcommand{\phas}[1]{\underline{#1}}
\newcommand{\RE}{\text{Re}}
\newcommand{\R}{\mathbb{R}}
\newcommand{\C}{\mathbb{C}}
\usepackage{xfrac}
\newcommand\independent{\protect\mathpalette{\protect\independenT}{\perp}}
\def\independenT#1#2{\mathrel{\rlap{$#1#2$}\mkern2mu{#1#2}}}

\renewcommand{\d}[1]{\ensuremath{\operatorname{d}\!{#1}}}

%% Box %%
\newcommand{\theor}[1]{\adjustbox{minipage=\linewidth-2\fboxsep-2\fboxrule,fbox}{\textsc{Théorème : }#1}}
\newcommand{\retenir}[1]{\adjustbox{minipage=\linewidth-2\fboxsep-2\fboxrule,fbox}{\textbf{\textit{\textsc{A retenir} : }}#1}}
\newcommand{\corollaire}[1]{\ \\\begin{tabular}{|c}
    \begin{minipage}{\textwidth}
      \textsc{Corollaire : } \textit{#1}
    \end{minipage}
    \end{tabular}}

%% Matter (main, front, back) %%%

\makeatletter

\newcommand\frontmatter{%
    \cleardoublepage
  %\@mainmatterfalse
  \pagenumbering{arabic}}

\newcommand\mainmatter{%
    \cleardoublepage
 % \@mainmattertrue
  \pagenumbering{arabic}}
  %\usepackage{lastpage}

\newcommand\backmatter{%
  \if@openright
    \cleardoublepage
  \else
    \clearpage
  \fi
 % \@mainmatterfalse
   }

\makeatother




%% ----------------------------------------------------------------
% package acronymes et glossaire
%% ----------------------------------------------------------------
\usepackage[savewrites,nopostdot,toc,acronym,nogroupskip]{glossaries}
\renewcommand{\glsnamefont}[1]{\textbf{#1}}
\setlength\LTleft{0pt}
\setlength\LTright{0pt}
\setlength\glsdescwidth{0.85\hsize}
\makeglossaries
%\loadglsentries{glossary}



%% ----------------------------------------------------------------
% package biblio
%% ----------------------------------------------------------------
\usepackage[
    %backend=biber, 
    natbib=true,
    style=numeric,
    sorting=none
]{biblatex}

\addbibresource{Bibliography.bib}






%% ----------------------------------------------------------------
% package appendixes
%% ----------------------------------------------------------------
\usepackage{appendix}
\usepackage{lscape}